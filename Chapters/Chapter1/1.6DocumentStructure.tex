\section{Document Structure}

The RASD is structured in the following five chapters:
\begin{itemize}
\item \textbf{Chapter 1 -  \textit{Introduction}:} It contains a general introduction to the problem of interest and a more detailed list of the goals of this project. The scope of the application is described thanks to the analysis of the phenomena involved.
A part relating to the definitions used in the document is also included.

\item \textbf{Chapter 2 -  \textit{Overall Description}:} It contains an overall description of the project including some possible scenarios of interest, the system class diagram and state diagrams.
The most important product functions necessary for the correct functioning of the application and all the assumptions on the domain are also described.

\item \textbf{Chapter 3 -  \textit{Specific Requirements}:} In this part the software product requirements are described in detail.
The description of the interfaces is included and the functional requirements are defined thanks to the use of UML diagrams.
This section also includes performance requirements, design constraints and software system attributes.

\item \textbf{Chapter 4 -  \textit{Formal Analysis using Alloy}:} It contains formal modeling of the software product using the Alloy tool and dedicated comments to better clarify each part of the modeling.

\item \textbf{Chapter 5 -  \textit{Effort Spent}:} It contains all the information regarding the hours of work required to create the document and the tasks assigned to each person in the team.
\end{itemize}

At the end of the document the bibliography is also included.
