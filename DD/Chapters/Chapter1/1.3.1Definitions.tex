\newpage
\subsection{Definitions}
\begin{center}
\setlength\tabcolsep{7pt}
\rowcolors{2}{white}{white!65!green2!50}
\renewcommand{\arraystretch}{2}
\begin{longtable}{|m{3.2cm}|m{8.3cm}|}
\caption{Definitions}\\
\hline
\endfirsthead
\endhead
\hline
\endlastfoot
\textit{The system} & The whole system to be developed \\
\textit{User} & A farmer/agronomist/policy maker who uses the application\\
\textit{Unregistered user} & A farmer/agronomist/policy maker who is not yet registered into the application\\
\textit{Application service} & Functionality offered by the system for certain users \\
\textit{ID Code} & Unique code that identifies each agronomist and policy maker\\
\textit{Policy maker} & The user of the application who decides about new policies for Telangana \\
\textit{Farmer} & The user of the application who owns or manages a farm\\
\textit{Address} & The address inserted by the farmer which corresponds to his/her farm's location. It is composed by city, zip code, street and number\\
\textit{Performance} & Indicator of the progress of a farmer's activity up to a certain date. Its value is calculated as numerical score \\
\textit{Score} & Performance rating computed by a function that depends on: type and quantity of harvested product, weather conditions, quantity of water consumed, soil moisture\\
\textit{Well performing farmer} & A farmer who has a score higher than a certain threshold\\
\textit{Bad performing farmer} & A farmer who has a score below a certain threshold\\
\textit{Agronomist} & The user of the application dealing with the management of a certain mandal\\
\textit{Mandal} & A local government area and  administrative division. Telangana is subdivided into districts which are themselves subdivided into mandals\\
\textit{Daily plan} & Application service that allows the agronomist to manage his/her daily work schedule. Specifically, it allows him/her to track and organize visits to farmers\\
\textit{Discussion forum} & Application service which a farmer can use to exchange ideas and opinions about a topic\\
\textit{Post} & Contributions of the participants to the discussion placed one below the other in sequence\\
\textit{Thread} & It is composed of the topic followed by the posts left by the various participants in the discussion\\
\textit{Help request} & Application service which a farmer can use to request for help to the agronomist or/and other well performing farmers\\
\textit{Help response} & Reply to a help request\\
\textit{Tier} & A machine or a group of machines\\
\textit{Layer} & Separate functional component that interact with other layers in some sequential and hierarchical way\\
\textit{Mockup} & Full-size model of a design or device, used for product presentations or other purposes\\
\end{longtable}
\end{center}