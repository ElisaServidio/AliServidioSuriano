\section{Testing Plan}

Once all components have been integrated, programmers could perform \textit{System Testing}.
The aim of this kind of testing if to verify functional and non-functional requirements.
Specifically, \textit{DREAM} will be subjct of the following tests:

\begin{itemize}
    \item \textbf{Functional Testing:} It ensures that the requirements and the specifications defined in the RASD are properly satisfied by the System. 
    \item \textbf{Performance Testing:} The main purpose is to identify and eliminate the performance bottlenecks in the software application affecting response time, utilization, throughput.
    \item \textbf{Load Testing:} It aims at detecting bugs such as memory leaks, mismanagement of memory and buffer overflows. It also identifies the maximum operating capacity of the application.
    \item \textbf{Stress Testing:} It verifies stability and reliability of software application. The goal is measuring software on its robustness and error handling capabilities under extremely heavy load conditions and ensuring that software doesn't crash under crunch situations. It even tests beyond normal operating points and evaluates how software works under extreme conditions.
\end{itemize}