\subsection{Domain Assumptions}

\begin{itemize}
    \item [\textit{D.1}] Each user who wants to use DREAM is needed to have a device connected to Internet 
    \item [\textit{D.2}] Data retrieved by external services (quantity of water consumed by each farmer, soil moisture and weather forecasts) are supposed to be accurate 
    \item [\textit{D.3}] A farmer does not forget to insert data regarding his/her production (type and quantity of product harvested)
    \item [\textit{D.4}] A farmer does not insert fraudulent information as input to  improve his/her performance score (e.g. higher quantity of product harvested)
    \item [\textit{D.5}] A farmer inserts in the system the correct address
    corresponding to his/her farm's location
    \item [\textit{D.6}] Each farmer is related to exactly one farm's address
    \item [\textit{D.7}] A farmer doesn't forget to solve a help request
    \item [\textit{D.8}]The external service used by the system to retrieve latitude and longitude of a farm using the address provided by a farmer is supposed to be accurate
    \item [\textit{D.9}] A unique ID code is given to each policy maker to be able to register
    \item [\textit{D.10}] A unique ID code is given to each agronomist to be able to register
    \item [\textit{D.11}] ID codes assigned to policy makers and agronomists from Telangana government are saved in a database
    \item [\textit{D.12}] An agronomist inserts in the system the correct mandal he/she is responsible of
    \item [\textit{D.13}] An agronomist updates the daily plan when scheduling a new visit
    \item [\textit{D.14}] An agronomist doesn't forget to confirm the daily plan by the end of the day specifying deviations correctly if needed
\end{itemize}

