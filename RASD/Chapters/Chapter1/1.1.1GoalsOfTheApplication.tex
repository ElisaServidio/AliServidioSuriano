\subsection{Goals of the Application}
\begin{itemize}
    \item \textit {G.1}: Allows policy makers to identify those farmers who are performing well
    \item \textit {G.2}: Allows policy makers to identify those farmers who are performing badly
    \item \textit {G.3}: Allows policy makers to verify the improvement of farmers who have been already helped by agronomist or good farmers
    \item \textit {G.4}: Allows farmers to visualize data relevant to them based on their location and type of production
    \item \textit {G.5}: {\color{red}  Allows farmers to insert in the system data about their production and any problem they face}
    \item \textit {G.6}: Allows farmers to request for help and suggestions by agronomists and other farmers
    \item \textit {G.7}: Allows farmers to create discussion forums with the other farmers
    \item \textit {G.8}: {\color{red}  Allows agronomists to insert the area they are responsible of}
    \item \textit {G.9}: {\color{red}  Allows agronomists to receive information about requests for help}
    \item \textit {G.10}: Allows agronomists to answer to requests for help from farmers
    \item \textit {G.11}: Allows agronomists to visualize data concerning weather forecasts in the area
    \item \textit {G.12}: Allows agronomists to visualize data concerning the best performing farmers in the area
    \item \textit {G.13}: Allows agronomists to visualize a daily plan to visit farms in the area
    \item \textit {G.14}: Allows agronomists to update a daily plan to visit farms in the area
    \item \textit {G.15}: Allows agronomists to confirm the execution of the daily plan at the end of each day 
    \item \textit {G.16}: Allows agronomists to specify the deviations from the daily plan at the end of the day

\end{itemize}