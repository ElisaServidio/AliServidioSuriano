\section{Purpose}

Telangana is the 11th largest and the twelfth-most populated state in India with a geographical area of 112,077 km2 and 35,193,978 residents (data from2011).
The economy of Telangana is mainly driven by agriculture, a sector which plays a pivotal role in all India’s economy: over 58\% of rural households depend on it as the principal means of livelihood, 80\% of whom are smallholder farmers with less than 2 hectares of farmland. More than a fifth of the small-holder farm households are below poverty. \\

Worldwide there are many threats to the agriculture sector.\\ World population is estimated to reach 9.7 billion by 2050, therefore food demand is expected to increase anywhere between 59\% to 98\% by 2050.Climate change is predicted to result in a 4\%-26\% loss in net farm income towards the end of the century. The COVID-19 pandemic has greatly exposed the vulnerabilities of marginalized communities, small holder farmers and the importance of building resilient food systems.\\

This calls for a revamp of the entire food supply chain to help bolster countries against shocks and challenges. For this reason, Telangana’s government aims to design, develop and demonstrate anticipatory governance models for food systems using digital public goods and community-centric approaches to strengthen data-driven policy making in the state. To achieve this goal, Telangana wants to partner with IT providers with the aim of acquiring and combining data concerning: weather forecasts, agriculture production (types and produced amount per product, amount of water used by each farmer), humidity of soil and information provided by the governmental agronomists. Acquiring and combining such data, the software system DREAM supports the work of three types of actors: policy makers, farmers, and agronomists.\\

DREAM allows Telangana’s policy makers to identify farmers who are performing well and those who are performing particularly badly. The first ones, especially the more resilient to meteorological adverse events, will receive special incentives and will be asked to provide useful best practices to the others. Moreover, the system will help policy makers to understand whether the steering initiatives carried out by agronomists with the help of good farmers produce significant results.
{\color{red} Thanks to the application the policy makers will be able to make decisions in the real world by analyzing and visualising data regarding farmers.}\\

On the other hand, farmers are allowed to visualize data relevant to them based on their location and type of production, such as weather forecasts and personalized suggestions (i.e. specific crops to plant or specific fertilizers to use). Farmers can exploit a sort of personal diary provided by the system \textcolor{red}{(aggiungere nome)}, in fact they can archive data daily regarding their production and any problem they face. They can also request for help and suggestion by agronomists and other farmers with whom they can also create discussion forums.\\

Eventually, agronomists are in charge of a certain mandal, which is a local government area and administrative division of Telangana. They can receive information about requests for help and answer them and they can visualize data concerning weather forecasts and the best performing farmers in the area. Furthermore, agronomists can visualize, up-date and confirm a daily plan to visit farms, assuming that all farms must be visited at least twice a year, but those that are under-performing should be visited more often, depending on the type of problem they are facing.

\subsection{Goals of the Application}
\begin{itemize}
    \item [\textit {G.1}] Allows policy makers to visualize and analyze data of farmers
    \item [\textit {G.2}] Allows policy makers to identify those farmers who are performing well
    \item [\textit {G.3}] Allows policy makers to identify those farmers who are performing badly
    \item [\textit {G.4}] Allows policy makers to verify the improvement of farmers who have been already helped by agronomist or good farmers
    \item [\textit {G.5}] Allows farmers to visualize data and suggestions relevant to them based on their location and type of production
    \item [\textit {G.6}] Allows farmers to insert in the system data about their production and any problem they face
    \item [\textit {G.7}] Allows farmers to request for help and suggestions by agronomists and other farmers
    \item [\textit {G.8}] Allows farmers to create discussion forums with the other farmers
    \item [\textit {G.9}] {\color{orange}  Allows agronomists to receive information about requests for help}
    \item [\textit {G.10}] Allows agronomists to answer to requests for help from farmers
    \item [\textit {G.11}] Allows agronomists to visualize data concerning farmers in the mandal
    \item [\textit {G.12}] Allows agronomists to visualize weather forecasts in the mandal
    \item [\textit {G.13}] Allows agronomists to visualize a daily plan to visit farms in the mandal
    \item [\textit {G.14}] Allows agronomists to update a daily plan to visit farms in the mandal
    \item [\textit {G.15}] Allows agronomists to confirm the execution of the daily plan at the end of each day 
    \item [\textit {G.16}] Allows agronomists to specify the deviations from the daily plan at the end of the day

\end{itemize}