\section{Scope}

The aim of the DREAM software product is to develop and adopt anticipatory governance models for food systems to strengthen data-driven state policy. \\
It takes care of the acquisition and management of all data collected in order to support the work of farmers, agronomists and policy makers.\\
The system aims to collect data not only from sensors located throughout the territory, but also from farmers. The analysis of the acquired data aims to improve the production of farmers.\\ Low-performing farmers are identified by policy makers and supported / helped by the best-performing ones.\\
Everything is supervised by agronomists who take care of their own geographical areas of competence.\\

To better understand all the phenomena involved, we distinguish them into two types according to the World and Machine paradigm [M. Jackson and P. Zane]. The World is the environment surrounding the system, while the Machine is the system itself.

\subsection{World and Shared phenomena}

\begin{center}
    
\setlength\tabcolsep{7pt}
\rowcolors{2}{white}{white!65!green2!50}
\renewcommand{\arraystretch}{2}
\begin{longtable}{ |m{7cm}|C{2.3cm}|C{1.6cm}|}
\caption{World and Shared phenomena\label{long}}\\
\hline
\endfirsthead
\endhead
\hline
\endlastfoot
\rowcolor{green2}
\textbf{Phenomenon} & \textbf{Who controls it?} & \textbf{Is it shared?}\\
\hline

A farmer signs up to the application or logs in if already registered & W & Y\\
Telangana government assigns to each agronomist his ID code & W & N\\
An agronomist registers into the application by using his ID code or logs in if already registered & W & Y\\
Telangana government assigns to each policy maker his ID code & W & N\\
\textcolor{green}{The system checks validity of ID code} & M & Y\\
A policy maker registers into the application by using his ID code or logs in if already registered & W & Y\\
The system processes the performance of a farmer (or more than one) & M & N\\
A policy maker visualizes the performance trend of a farmer (or more than one) over time & W & Y\\
A policy maker visualizes the performance score of a farmer (or more than one) & W & Y\\
\noalign{\global\arrayrulewidth=0.3mm}
\arrayrulecolor{gray}\hline
The system notifies a farmer who is not performing well that it would be a good idea to request for help to an agronomist or a well performing farmer & M & Y\\
The system notifies a farmer that he improved his performance and he is entitled to receive an incentive & M & Y\\
The system retrieves longitude and latitude from the address of a farmer's land & M & N\\
The system sends personalized suggestions to a farmer based on computed data & M & Y\\
\textcolor{green}{The system retrieves weather forecasts, humidity of soil and amount of water used corresponding to a certain position} & M & Y\\
A farmer visualizes relevant information which are: weather forecasts, soil moisture and personalized suggestions and visits & W & Y\\
A farmer insert in the system the production of a certain period and the corresponding relevant data &  W & Y\\
\textcolor{green}{A farmer makes a help request} & W & Y\\
\textcolor{green}{The system assigns the help request to an agronomist and/or a well performing farmer} & M & N\\
A farmer opens a thread about a topic on the dedicated forum & W & Y\\
A farmer calls the agronomist to reschedule a visit & W & N\\
The system associates each farmer to the corresponding mandal & M & N\\
\noalign{\global\arrayrulewidth=0.3mm}
\arrayrulecolor{gray}\hline
An agronomist inserts in the system the mandal he is responsible of & W & Y\\
An agronomist receives a help request from a farmer & M & N\\
An agronomist replies to the help requests sent to him by farmers & W & Y\\
An agronomist calls a farmer to cancel a visit & W & N\\
An agronomist visualizes data about farmers of his mandal & W & Y\\
An agronomist visits a farm belonging to his mandal & W & N\\
An agronomist visualizes weather forecasts on the app & W & Y\\
The system creates the agronomist's daily plan, guaranteeing at least two visits per year for each farmer & M & N\\
An agronomist visualizes his daily plan & W & Y\\
An agronomist updates his daily plan & W & Y\\
An agronomist carries out his daily plan & W & N\\
An agronomist confirms his daily plan execution & W & Y\\
An agronomist specifies deviations from his daily plan at the end of the day & W & Y\\

\end{longtable}

\end{center}